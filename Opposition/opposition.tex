\documentclass{article}

\setlength\parindent{0pt}

\title{Opposition report: Comparative study on road and lane detection in mixed criticality systems}
\author{Emil Hjelm}

\begin{document}

\maketitle

The report is well written, interesting and quite enjoyable to read with not too many typos. The author definitely succeeds in interesting the reader in the topic and the author seems motivated by the task.\\

The report is well structured with a clear goal. There is a clear direction and the different sections are well connected with each other. The background study built up the required knowledge to solve the engineering problem and answer the research question. There are clear links in the implementation part to the background study. The background study is of appropriate magnitude.\\

The figures and tables used are of good quality and provide helpful information.\\

The general formatting of the report is good.\\

The work undertaken during the project achieved its purpose and goals and the report reflects very well the engineering work.\\

%Does the report reflect the engineering work undertaken during the project?\\

%Does the author use a scientific and systematic approach?\\

The gathering of results and general approach was scientific and systematic. The methods used were consciously chosen. (Gathering execution times of parts of the algorithm, system identification etc.)\\

%Has the author made a conscious choice of method based on their problem?\\

%Are there clear links to relevant theories from the background study?\\

%1. General impressions\\

%How valuable is the thesis work to the industrial supervisor / employer/research project?\\

The thesis seems to be of high value to the larger research project and the commissioner. \\

%Has the author contributed something new in their chosen field?\\

The author has not really contributed something new in their chosen field. (But that's asking for a lot, no?)\\%The mixed criticality approach is new but not 

%3 Topic, problem area\\

%Is the chosen topic and problem area of interest to study?\\

The topic area is of great interest. Both in industry, but it also seems to fascinate people outside of the engineering field.\\

%4 Problem discussion and purpose\\

%Does the author succeed in interesting the reader in the topic?\\

%Is the author motivated by the task?\\

%Is the purpose clearly described?\\

The purpose is clearly described.\\

%Is the objective achievable?\\

%Is it clear from the beginning which issues the author is addressing?\\

It is not entirely clear how the author will address the issue of "mixed criticality" from the beginning, all other issues are clearly described.\\

%Do you feel that any important issue has been overlooked?\\

%5 Boundaries\\

%Does the work have reasonable and clear boundaries?\\

The author has quite clear boundaries, one boundary needs to be remade though: "Constrained to the Xilinx Zynq-7000". Since the Raspberry Pi was used, this boundary is not kept.\\

%Has the author motivated the limitations/boundaries made for the project?\\

%6 Methodology\\

%Is it clear which method the author has chosen?\\

%What motivated the choice of method? Was the choice deliberate?\\

%Are there feasible alternatives to the chosen method? Has the author described and discussed these?\\

%How has the author handled and discussed verification and validation, or reliability and\\ validity?\\

%References\\

%What are the references the author has chosen to use?\\
The author has used solid references. They are presented in a very consistent manner at all times when references are needed.\\

%Are the references presented in a consistent manner?\\

%8 The Frame-of-reference\\

%How is the information-gathering done?\\

%Is the presentation of the Frame-of-reference, ie the knowledge base for work, logical and clear with regard to the purpose?\\

%Is the magnitude of the frame of reference appropriate?\\

%9 Analysis\\

%How has the author made use of the knowledge base in the frame of reference?\\

%How has the author used the knowledge accumulated in the frame of reference for interpreting the results?\\

%10 Results and Conclusions\\

%Are the findings and conclusions sustainable?\\

%Has the author made any suggestions or recommendations to their employer?\\

The author has made multiple suggestions for future work for its employer, all of which are good continuations on the authors work.\\

%Does the work undertaken in the project achieve the purpose?\\

%11 Discussion\\

%Is there any evaluation of the methods used?\\

%Does the author discuss the limitations of the work performed?\\

%Are there any specified proposals for future work?\\

%12 Outline and logic\\

%Do you see any common thread (are the different sections logically connected to each other)?\\

%Are any sections given too much or too little space?\\

%Is there anything you feel the author is missing?\\

One thing I feel could be added is an comparison between different algorithms in the implementation. Comparison is made in the literature study, but it would have been interesting to see some differences between different algorithms in the implementation.

%13 Language and technical execution\\

%Does the thesis contain typographical errors, faulty sentence construction or other similar defects?\\

%Are the sections divided logically?\\

%Are the table of contents, headings and references consistent and accurate?\\

%Are the figures and tables (and the text for the figures and tables) well presented?\\

%14 Honesty and critical distance\\

%Is it easy for the reader to discern what is taken from literature and other sources and what is the authors own opinion?\\

%Has the author maintained a critical distance to theories, established knowledge and the conclusions drawn?\\

\end{document}
