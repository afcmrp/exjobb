\noindent \begin{tabular*}{1.0\textwidth}{|@{} p{0.9835\textwidth}|}
\hline
\noindent \begin{tabular*}{1.0\textwidth}{p{0.97\textwidth}}
\textcolor{white}{.}\\[-10pt]
\end{tabular*}
\noindent \begin{tabular*}{1.0\textwidth}{p{0.24\textwidth} p{0.69\textwidth}}
\multirow{3}{*}{\includegraphics[scale=0.18]{./img/KTH_Logotyp_RGB_2013}} & \begin{center}Master Thesis MMK2017:Z MDAZZZ\end{center}\\[-20pt]
& \begin{center}Safety-critical Control in Mixed Criticality Embedded Systems \end{center}\\[-20pt]
& \begin{center}Emil Hjelm \end{center}\\ 
\end{tabular*}
\noindent \begin{tabular*}{1.0\textwidth}{p{0.24\textwidth}|p{0.33\textwidth}|p{0.33\textwidth}}
\hline
{ \footnotesize Approved:} & { \footnotesize Examiner:} & { \footnotesize Supervisor:}\\
(datum) & Martin Törngren & Bengt Eriksson \\
\hline
& { \footnotesize Uppdragsgivare:} & { \footnotesize Kontaktperson:}\\
& Alten & Detlef Scholle \\ \hline
\end{tabular*}
\end{tabular*}
\textcolor{white}{.}\\[0.5cm]
{\Large Abstract}\\
\textcolor{white}{.}\\
\label{sec:abstract}

%This section will be the abstract of the report in English.
Modern automotive systems contain a large number of Electronic Control Units, each controlling a specific system of a specific criticality level. To increase computational efficiency it is desired to combine multiple applications into fewer ECUs, this leads to mixed criticality embedded systems. The assurance of safety critical applications not being affected by non-critical applications on the same system is crucial. 

\setcounter{page}{1}
\vspace{0.25cm}
%\keywords{mixed criticality embedded systems, safety critical control}